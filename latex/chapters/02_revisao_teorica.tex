\chapter{Revisão teórica}
\label{ch:revisão}

Todas as definições e teoremas são retirados diretamente ou
readaptados para melhor clareza de \cite{diestel2025},
\cite{polya2010}, \cite{jensen2010} e \cite{gersting1995}. Como os
algoritmos são implementados em inglês, apresentaremos o
correspondente ao termo em inglês.

\begin{mydef}[Grafo]
  Um \emph{grafo} (\textit{graph}) é uma estrutura $G := (V, A)$
  tal que $A \subseteq V^2$ e $V$ é um conjunto de um tipo qualquer.
  Os elementos de $V$ são denominados vértices (\textit{nodes}) e os
  elementos de $A$ são denominados de arestas
  (\textit{edges}). O jeito tradicional de visualizar um
  grafo é como uma figura composta de bolas e setas:
  \begin{figure}[h]
    \centering
    \begin{tikzpicture}
      \GraphInit[vstyle=normal]
      \tikzset{EdgeStyle/.style={->}}
      \Vertices{circle}{1,2,3,4}
      \Edges(1,2,3)
      \Edge(1)(4)
    \end{tikzpicture}
    \caption{Um grafo com $V := \{1,2,3,4\}$ e $A :=
    \{(1,2),(1,4),(2,3)\}.$}
    \label{fig:graph1}
  \end{figure}
  \FloatBarrier
\end{mydef}

\begin{mydef}[Grafo rotulado]
  Dizemos que um grafo $G := (V, A)$ é \emph{rotulado} quando há
  informações de identificação (rótulos) nos vértices do grafo. Tais
  rótulos podem ser numéricos ou alfabéticos.
\end{mydef}

\begin{mydef}[Grafo ponderado]
  Dizemos que um grafo $G := (V, A, c)$ é \emph{ponderado}
  (\textit{weighted}) quando $c$ é uma função $A \to \mathbb{R}$,
  onde $c_\alpha$ representa o custo de atrevessar uma aresta $\alpha \in A$.
\end{mydef}

\begin{mydef}[Ordem e Tamanho]
  O número de vértices de um grafo $G$ é chamado de \emph{ordem}
  (\textit{order}) e é denotado por $|G|$ -- o número de arestas é chamado de
  \emph{tamanho} (\textit{size}) e é denotado por $||G||$. Por exemplo, na
  Figura~\ref{fig:graph1}, $|G| = 4$ e $||G|| = 3$.
\end{mydef}

\begin{mydef}[Adjacência]
  Dizemos que um vértice $v$ é \emph{adjacente}, ou \emph{vizinho},
  de um vértice
  $u$ (\textit{neighbor}) se somente se $(u,v) \in A$, também,
  denotaremos $(u,v)$ como $uv$. Visualmente,
  enxergamos isso como:
  \begin{figure}[h]
    \centering
    \begin{tikzpicture}
      \GraphInit[vstyle=normal]
      \tikzset{EdgeStyle/.style={->}}
      \Vertices{circle}{v,u}
      \Edge(u)(v)
    \end{tikzpicture}
    \caption{Um grafo com $V := \{u,v\}$ e $A :=
    \{(u,v)\}.$}
  \end{figure}
  \FloatBarrier
\end{mydef}

\begin{mydef}[Conjunto de adjacentes]
  Num grafo $G$, o conjunto de todos os vértices adjacentes de $u$
  (\textit{neighbors}) é denotado por $A_G(u) := \{v \in V\, |\, uv \in
  A\}$. Já o conjunto de todos os vértices que em que $v$ é adjacente
  será denotado por $\bar{A}_G(u) := \{ v \in V\, |\, vu \in A\}$.
\end{mydef}

\begin{mydef}[Grau de um vértice]
  O \emph{grau de um vértice} $v$ (\textit{node degree}) é o valor
  correspondente da soma $|A_G(v)| + |\bar{A}_G(v)|$. Também
  denotamos $|A_G(v)|$ como $d^+(v)$, $|\bar{A}_G(v)|$ como $d^-(v)$ e
  sua soma como $d(v)$.
\end{mydef}

\begin{mydef}[Grafo não direcionado]
  Dizemos que um grafo $G$ é \emph{não direcionado}
  (\textit{undirected}) se somente se $A$ é simétrico, ou seja, se
  $uv \in A$ então $vu \in A$. O nome não direcionado vem da ideia de
  que os grafos que viemos discutindo até agora são denomidados de
  \emph{direcionados}, ou simplesmente \emph{dígrafos}. Na
  literatura é comum apresentar grafo não direcionado como grafo e
  depois o direcionado como dígrafo, resolvemos inverter a ordem pois
  assim se traduz melhor nas representações de grafos que vamos
  implementar. Um grafo não direcionado pode ser visualizado sem a
  ponta das setas:

  \begin{figure}[h]
    \centering
    \begin{tikzpicture}
      \GraphInit[vstyle=normal]
      \Vertices{circle}{c,b,a}
      \Edges(a,b,c)
    \end{tikzpicture}
    \caption{Um grafo não direcionado com $V := \{a,b,c\}$ e $A :=
    \{ab,ba,bc,cb\}$}
    \label{fig:ungraph1}
  \end{figure}

  Também é comum omitir a simetria das arestas se pelo contexto for claro que
  está sendo tratado de um grafo não direcionado, na
  Figura~\ref{fig:ungraph1}, o
  conjunto de arestas $A$ seria escrito como $\{ab,bc\}$.
\end{mydef}

\begin{mydef}[Caminho]
  Um \emph{caminho} (\textit{path}) é um grafo $C := (V, A)$ que tem a forma:
  \begin{displaymath}
    V := \{x_0, x_1,...,x_k\} \qquad A := \{x_0x_1,x_1x_2,...,x_{k-1}x_k\}
  \end{displaymath}
  Dizemos que $C$ é um caminho de $x_0$ a $x_k$. Normalmente nos
  referimos ao caminho como a sequência dos seus vértices, $x_0x_1...x_k$.
\end{mydef}

\begin{mydef}[Conectividade]
  Dizemos que um grafo $G$ é \emph{conexo} (\textit{connected}) se somente
  se para quaisquer dois vértices $u$ e $v$, existe um caminho entre eles.
\end{mydef}

\begin{mydef}[Ciclo]
  Dizemos que um grafo $G:= (V, A)$ contém um \emph{ciclo}
  (\textit{cycle}) quando há um caminho possível do vértice $v_i$ até
  ele próprio sem passar mais de uma vez por vértices intermediários.
  Quando não há ciclos no grafo, dizemos que ele é \emph{acíclico}.
\end{mydef}

\begin{mydef}[Caminho Euleriano]
  Um \emph{Caminho Euleriano} (\textit{Eulerian Path}) em um grafo $G
  := (V, A)$ é um caminho que usa cada uma das arestas de $G$ exatamente 1 vez.
  \begin{figure}[h]
    \centering
    \begin{tikzpicture}
      \GraphInit[vstyle=normal]
      \tikzset{EdgeStyle/.style={->}}
      \Vertices{circle}{1,2,3,4}
      \Edges(1,2,3,4,1,3)
    \end{tikzpicture}
    \caption{Um grafo com $V := \{1,2,3,4\}$ e $A :=
      \{(1,2),(2,3),(3,4),(4,1),(1,3)\}$.
      O caminho $1 \to 2 \to 3 \to 4 \to 1 \to 3$ é um exemplo de
    caminho euleriano.}
    \label{fig:eulerian-path}
  \end{figure}
  \FloatBarrier
\end{mydef}

\begin{mytherm}[Teorema dos Caminhos Eulerianos]
  Existe um caminho euleriano em um grafo não direcionado $G := (V,
  A)$ se e somente se existir exatamente 0 ou 2 vértices de grau
  ímpar no grafo. Existe um caminho euleriano em um grafo direcionado
  $G := (V_2, A_2)$ se e somente se existe no máximo 1 vértice com
  grau de saída maior que grau de entrada por 1 e no máximo 1 vértice
  com grau de entrada menor que grau de saída por 1.
\end{mytherm}

\begin{mydef}[Ciclo Euleriano]
  Um \emph{Ciclo Euleriano} (\textit{Eulerian Cycle}) em um grafo $G
  := (V, A)$ é um caso particular do Caminho Euleriano onde o Caminho
  inicia e termina no mesmo vértice $v$.
\end{mydef}

\begin{mydef}[Grafo Euleriano]
  Dizemos que um grafo $G := (V, A)$ é um \emph{Grafo Euleriano}
  (\textit{Eulerian Graph}) se o grafo contém um Ciclo Euleriano.
  \begin{figure}[h]
    \centering
    \begin{tikzpicture}
      \GraphInit[vstyle=normal]
      \tikzset{EdgeStyle/.style={->}}
      \Vertices{circle}{A,B,C,D}
      \Edges(A,B,C,A,B,D,A)
      \draw[->, bend right=20] (A) to (B);
    \end{tikzpicture}
    \caption{Um grafo euleriano com $V := \{A, B, C, D\}$ e $A :=
      \{(A, B), (A, B), (B, C), (C, A), (B, D), (D,A)\}$.
    Exemplo de ciclo euleriano: $A \to B \to C \to A \to B \to D \to A$.}
    \label{fig:eulerian-graph}
  \end{figure}
  \FloatBarrier
\end{mydef}

\begin{mytherm}[Teorema do Grafo Euleriano]
  Um grafo conexo não orientado $G_1 := (V_1, A_1)$ é Euleriano se e
  somente se todos os vértices tem grau par. Um grafo conexo
  orientado $G_2 := (V_2, A_2)$ é Euleriano se e somente se todos os
  vértices tem o mesmo grau de entrada e saída.
\end{mytherm}

\begin{mydef}[Caminho Mais Curto]
  Dizemos que $C := (V, A)$ é o \emph{caminho mais curto}
  (\textit{shortest path}) entre dois
  vértices $u$ e $v$ sse para todo caminho $C' := (V', A')$ entre $u$ e $v$:
  \begin{equation}
    \sum_{\alpha \in A} c_\alpha \leq \sum_{\alpha \in A'} c_\alpha
  \end{equation}
\end{mydef}

\begin{mydef}[Subgrafo]
  Dizemos que $G':= (V', A')$ é \emph{subgrafo} (\textit{subgraph})
  de um grafo $G:= (V, A)$ quando $G'$ consiste em um subconjunto de
  vértices e arestas do grafo original, mas preservando a adjacência
  entre os vértices.
  \begin{figure}[h]
    \centering
    \begin{tikzpicture}
      \GraphInit[vstyle=normal]
      \tikzset{EdgeStyle/.style={->}}

      \Vertices{circle}{A,B,C,D,E}

      % Arestas do grafo G
      \Edges(A,B,C,D,E,A)
      \Edges(B,D)
    \end{tikzpicture}

    \caption{Grafo $G$ com $V = \{A,B,C,D,E\}$ e
    $A = \{(A,B), (B,C), (C,D), (D,E), (E,A), (B,D)\}$.}
    \label{fig:subgraph-g}
  \end{figure}

  \begin{figure}[h]
    \centering
    \begin{tikzpicture}
      \GraphInit[vstyle=normal]
      \tikzset{EdgeStyle/.style={->}}

      % Subconjunto de vértices
      \Vertices{circle}{A,B,D,E}

      % Arestas do subgrafo G'
      \Edges(A,B,D,E,A)
    \end{tikzpicture}

    \caption{Subgrafo $G'$, com $V' = \{A,B,D,E\}$ e
    $A' = \{(A,B), (B,D), (D,E), (E,A)\}$, preservando as adjacências de $G$.}
    \label{fig:subgraph-gp}
  \end{figure}

  \FloatBarrier
\end{mydef}

\begin{mydef}[Árvore]
  Dizemos que um grafo $G := (V, A)$ é uma \emph{árvore}
  (\textit{tree}) quando $G$ é acíclico e conexo. Além disso, dizemos
  que a árvore é enraizada quando fixamos um vértice como \emph{raiz}
  (\textit{root}) ou não-enraizada quando não há raiz.
  \begin{figure}[h]
    \centering
    \begin{tikzpicture}
      \GraphInit[vstyle=normal]
      \tikzset{EdgeStyle/.style={->}}

      % Posições manuais para formar uma árvore em níveis
      \Vertex[x=0, y=2]{A}
      \Vertex[x=-1.5, y=0.8]{B}
      \Vertex[x=1.5, y=0.8]{C}
      \Vertex[x=-2.5, y=-0.8]{D}
      \Vertex[x=-0.5, y=-0.8]{E}
      \Vertex[x=1.5, y=-0.8]{F}

      \Edges(A,B)
      \Edges(A,C)
      \Edges(B,D)
      \Edges(B,E)
      \Edges(C,F)
    \end{tikzpicture}

    \caption{Árvore enraizada em $A$.}
  \end{figure}

  \FloatBarrier
\end{mydef}

\begin{mydef}[Árvore Geradora]
  Dizemos que $T := (V, A')$ é uma \emph{árvore geradora} de $G :=
  (V, A)$ quando $T$ é um subgrafo de $G$ conexo e acíclico. É uma
  forma de conectar todos os vértices de um grafo sem ciclos.
  \begin{figure}[h]
    \centering
    \begin{tikzpicture}
      \GraphInit[vstyle=normal]
      \tikzset{EdgeStyle/.style={->}}

      \Vertex[x=0, y=2]{A}
      \Vertex[x=-2, y=1]{B}
      \Vertex[x=2,  y=1]{C}
      \Vertex[x=-1, y=0]{D}
      \Vertex[x=1,  y=0]{E}

      \Edges(A,B)
      \Edges(B,C)
      \Edges(C,A)
      \Edges(B,D)
      \Edges(C,E)
      \Edges(D,E)
    \end{tikzpicture}

    \caption{Grafo $G$, note a existência do ciclo $A \to B \to C \to A$}
  \end{figure}

  \begin{figure}[h]
    \centering
    \begin{tikzpicture}
      \GraphInit[vstyle=normal]
      \tikzset{EdgeStyle/.style={->}}

      \Vertex[x=0, y=2]{A}
      \Vertex[x=0, y=1]{B}
      \Vertex[x=-1.5, y=0]{C}
      \Vertex[x=1.5, y=0]{D}
      \Vertex[x=-1.5, y=-1]{E}

      \Edge(A)(B)
      \Edge(B)(C)
      \Edge(B)(D)
      \Edge(C)(E)
    \end{tikzpicture}

    \caption{Árvore geradora $T$ de $G$. Note que não há mais ciclos.}
  \end{figure}
  \FloatBarrier
\end{mydef}

\begin{mydef}[Árvore Geradora Mínima]
  Dizemos que $T := (V, A')$ é a árvore geradora mínima (ou AGM) de
  um grafo ponderado $G := (V, A, c)$ quando é a árvore geradora de
  menor custo dentre todas as geradoras de $G$.
\end{mydef}
